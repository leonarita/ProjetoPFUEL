
% Definições para glossario

% ATENCAO o SHARELATEX GERA O GLOSSARIO/LISTAS DE SIGLAS SOMENTE UMA VEZ
% CASO SEJA FEITA ALGUMA ALTERAÇÃO NA LISTA DE SIGLAS OU GLOSSARIO É NECESSARIO UTILIZAR A OPÇÃO :
% "Clear Cached Files" DISPONIVEL NA VISUALIZAÇÃO DOS LOGS
% ---
% https://www.sharelatex.com/learn/Glossaries

\newglossaryentry{API} {
    name=API,
    plural= {APIs},
    description={Uma Interface de Progamação de Aplicação (\emph{Application Programming Interface}), são pontos que a
    aplicação expoõe para permitir que usuários ou serviços externos
    executem tarefas dentro da aplicação.}
}

\newglossaryentry{backend} {
    name={back-end},
    description={Um sistema \emph{back-end} é aquele que encontra na
    camada de servidor, em uma aplicação de duas camadas. Sua
    principal função é fornecer informações e capacidade de
    processamento a aplicação cliente.}
}

\newglossaryentry{build} {
    name={build},
    description={Versão compilada de um software}
}

\newglossaryentry{cdn}{
  name={CDN},
  description={Content distribution network é uma ferramenta de
    distribuição de conteúdo que mantém cópias dos arquivos estáticos
    de um domínio em diversas partes do mundo. O uso desta ferramenta
    diminui o tempo de latência das requisições HTTP, uma vez que o
    conteúdo pode ser acessado de um servidor mais próximo de onde o
    usuário se encontra.}
}

\newglossaryentry{crud} {
    name=CRUD,
    plural= {CRUDs},
    description={Create Retrieve Update Delete - Interface de usuário para manutenção em banco de dados que consiste somente nas operações básicas do banco de dados, sem nenhum tipo de inteligencia adicional de forma a facilitar ao usuário com o tratamento de algum processo}
}

\newglossaryentry{daily} {
    name=daily,
		plural= {dailies},
    description={Cerimônia diária do Scrum, é uma reunião de geralmente 15 minutos com o objetivo de alinhar o que será feito no dia e se existe algum impedimento.}
}

\newglossaryentry{deploy} {
    name=deploy,
    plural= {deploys},
    description={Processo pelo qual uma aplicação é compilada, gerando
      um arquivo executável ou imagem de Docker, e estes são enviados a
      um ambiente de homologoação ou de produção.}
}

\newglossaryentry{feedRSS} {
	name={Feed RSS},
	plural={feeds RSS},
	description={Recurso de distribuição de conteúdo em tempo real baseado na linguagem XML. Tecnologia que permite que os usuários de um blog acompanhem suas atualizações por meio de um software, website ou browser agregador.}
}

\newglossaryentry{framework} {
    name=framework,
    plural= {frameworks},
    description={Conjunto de ações e estratégias que possuem um ojeto e objetivo em específico;
    Na programação, é um conjunto de pacotes e bibliotecas que abstarai
    alguma função complexa, geralmente de nível mais baixo, e sobre o
    qual uma aplicação pdoe ser contruída. Ex: o framework web Spring
    asbtrai a lógica de implementação de um servidor, auxiliando a
    criação de endpoints, rotas e processamento de HTTP.}
}

\newglossaryentry{frontend} {
    name={front-end},
    description={Um sistema \emph{front-end} é aquele que encontra na
    camada cliente, em uma aplicação de duas camadas. Sua
    principal função, no escopo deste projeto, é atuar como interface
    gráfica para o usário, coletar dados e enviá-los para o \emph{back-end}.}
}

\newglossaryentry{GUI} {
    name=GUI,
    description={Interface gráfica de usuário é uma forma visual de se
    apresentar dados e coletar interações com o usuário, em oposição a
    linha de comando, que funciona por texto apenas.}
}

\newglossaryentry{hostname} {
    name=hostname,
    description={Nome que identifica o servidor ou máquina na rede.}
}

\newglossaryentry{jpa} {
    name=JPA,
    plural= {JPAs},
    description={Java Persistence API - Especificação do Java, usado para persistir dados de um objeto Java para um banco de dados relacional, atuando como uma ponte}
}

\newglossaryentry{kanban} {
    name=Kanban,
    description={Framework de gestão visual de fluxo de produção, que consiste no uso de cartões coloridos que separam as diferentes fases do desenvolvimento das tarefas (to do, doing, done).}
}

\newglossaryentry{load-balancing}{
  name={load balancing},
  description={Processo de equilibrar requisições que chegam ao
    back-end entre as diversas instâncias que o compõe, evitando que
    uma instância seja sobrecarregada com requisições.}
}

\newglossaryentry{ORM} {
    name=ORM,
    plural= {ORMs},
    description={Object-relational mapping - é uma técnica que permite consultar 	e manipular dados de um banco de dados usando um paradigma orientado a 				objetos}
}

\newglossaryentry{product-backlog} {
    name={Product Backlog},
    description={Lista de todos os requisitos e funcionalidades desejadas para um produto.}
}

\newglossaryentry{proxy}{
  name={proxy},
  description={Intermediário entre um cliente e um servidor. Quando
    usado como ``proxy reverso'', significa um ponto pelo qual passam
    todas as requisições que chegam a uma subrede.}
}

\newglossaryentry{pull} {
    name=pull request,
    plural={pull requests},
    description={Componente do ciclo de desenvolvimento do ponto de
      vista de sistemas de controle de versão, especifamente da
      ferramenta git. O pull request é uma forma de proteger o código
      primcipal do projeto, exigindo que mudanças sejam feitas em
      ramos distintos do código, e passem por um processo de revisão
      antes de serem incluidas no ramo principal.}
}

\newglossaryentry{repository}{
  name={Repository},
  plural= {Repositories},
  description={Classes que possuem métodos para controle e acesso ao banco de dados em uma aplicação Java}
}

\newglossaryentry{REST} {
    name=REST,
    description={A Tranfêrencia por Representação de Estado é uma forma
    de se transferir dados na qual os atributos de um objeto (seu
    estado) são serializados em um arquivo de formato específico, e
    o objeto pode ser reconstituído na aplicação que recebe o arquivo.}
}

\newglossaryentry{review} {
    name={Sprint Review},
    description={Cerimônia do Scrum que ocorre no fim da Sprint
    com o objetivo de validar as entregas daquele período.}
}

\newglossaryentry{retrospective} {
    name={Sprint Retrospective},
    description={Cerimônia do Scrum que ocorre no fim da Sprint com
    o objetivo de avaliar os pontos positivos e negativos
    da Sprint, agrupando possibilidades de melhorias
    para a próxima Sprint.}
}

\newglossaryentry{rollback} {
    name={rollback},
    plural={rollbacks},
    description={Reversão de um recurso modificado para uma versão predecessora}
}

\newglossaryentry{solid} {
    name=SOLID,
    description={São 5 princípios da programação orientada a objetos que visam facilitar o desenvolvimento de software}
}

\newglossaryentry{scrum} {
    name=Scrum,
    description={Framework ágil utilizado em desenvolvimentos iterativos
    e incrementais.}
}

\newglossaryentry{sprint} {
    name=Sprint,
plural={Sprints},
    description={Período de tempo limitado dentro do Scrum onde uma quantidade de histórias de usuário incrementáveis são desenvolvidas.}
}

\newglossaryentry{sprint-backlog} {
    name={Sprint Backlog},
    description={Lista das histórias de usuário que serão desenvolvidas durante a Sprint a ser iniciada.}
}

\newglossaryentry{planning} {
    name={Sprint Planning},
    description={Cerimônia do Scrum que ocorre antes do início
    de uma nova Sprint com o objetivo de planejar quais histórias
    de usuário serão desenvolvidas naquela iteração.}
}

\newglossaryentry{ssl} {
    name={SSL},
    description={Secure Sockets Layer - forma de segurança digital que permite que um site e um navegador se comuniquem de forma criptografada.}
}

\newglossaryentry{url} {
	name=URL,
	plural= {URLs},
	description={Uniform Resource Locator - inidica onde um recurso está 				localizado na rede, seja essa rede a internet ou a intranet}
}


% Normalmente somente as palavras referenciadas são impressas no glossario, portanto é necessário referenciar utilizando \gls{identificação}
