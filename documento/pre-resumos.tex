% ---
% RESUMOS
% ---

% resumo em português
\setlength{\absparsep}{18pt} % ajusta o espaçamento dos parágrafos do resumo
\begin{resumo}
 Comprar é uma atividade comum na rotina das pessoas e afeta a organização do tempo e do dinheiro. O tempo é gasto elaborando listas, revisando quanto foi gasto e com o quê, e às vezes comprando algo que outra pessoa já comprou (no caso de compras coletivas). Na perspectiva financeira, é essencial saber o quanto está sendo gasto no decorrer de uma compra, saber onde os produtos são mais baratos e onde se gastou mais. Considerando-se esse cenário, foi desenvolvido o aplicativo Lixt para gerenciamento de listas de compras que podem ser compartilhadas. O Lixt oferece funcionalidades que permitem o mapeamento eficiente das compras através de um histórico de informações e também disponibiliza estatísticas de consumo ao usuário. O compartilhamento das listas proporciona mais organização e agilidade nas compras coletivas. Para se desenvolver este sistema, aplicou-se o método ágil Scrum e a ferramenta Kanban. A implementação técnica utilizou o \gls{sgbd} MySQL. O back-end desenvolvido em Java com base na arquitetura \gls{api} \gls{rest} e os \glspl{framework} Spring e Hibernate. A aplicação foi hospedada na plataforma Microsoft Azure. Para o desenvolvimento da interface de usuário (\gls{frontend}), utilizou-se a biblioteca JavaScript React Native. A solução desenvolvida entrega ao usuário um conjunto de funcionalidades único que nenhum dos concorrentes analisados abrange completamente. Além disso, as funcionalidades implementadas auxiliam o usuário durante todo o processo de uma compra: planejamento, execução e análise de gastos.

 \textbf{Palavras-chaves}: lista de compras. compras. organização financeira.
\end{resumo}

% resumo em inglês
\begin{resumo}[Abstract]
 \begin{otherlanguage*}{english}
Shopping is a common activity in people's routine and affects the organization of time and money. Time is spent making lists, reviewing how much was spent and on which items, and sometimes buying something someone else has already bought. From a financial perspective, it is essential to know how much is being spent during a purchase, to know where the products are cheaper and where the most amount of money was spent. Considering this scenario, the Lixt application was developed for managing shopping lists that can be shared with other users. Lixt offers features that allow efficient mapping of purchases through a history of information such as date, establishment and amount spent on products; it also displays consumption statistics such as in which categories there are more expenses (food, cleaning, etc.). The lists sharing provides more organization and agility in collective purchases. To develop the system, activities were managed with the agile scrum methodology and the kanban tool, the technical implementation used the MySQL relational database with the back-end developed in Java based on the \gls{rest} \gls{api} architecture and the Spring and Hibernate frameworks, all hosted on the Microsoft Azure platform. For the user interface (\gls{frontend}) the JavaScript library React Native was used. Through technology, the presented solution allowed an improvement in the users' shopping experience, reducing the time spent to prepare lists and the effort that would be necessary to map expenses manually.
 
   \vspace{\onelineskip}

   \noindent 
   \textbf{Keywords}: shopping list. purchases. financial organization.
 \end{otherlanguage*}
\end{resumo}