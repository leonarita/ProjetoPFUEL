Nossa equipe criou o blog chamado Equipe \gls{tgt}, onde semanalmente um dos membros da equipe criava uma postagem referente a como foi a semana anterior. Nesse apêndice estão as publicações que colocamos no blog.\\

\textbf{Título}: 1° Semana do projeto

\textbf{Data da publicação}: 24 de maio de 2021

\textbf{Autora}: Gabriely

A primeira semana da equipe \gls{tgt} foi marcada por diversas reuniões de alinhamento. Essas conversas foram de extrema importância para a sondagem de conhecimentos técnicos que cada um do grupo possui e para o brainstorm da aplicação a ser desenvolvida. Foram realizados dois \textit{checkpoints} essenciais: o de distribuição de tarefas referentes a proposta que deve ser entregue no dia 25/05/2021 e o de validação de andamento da proposta, para garantir que tudo está devidamente feito, formatado, e adequado para a apresentação. A cada dia o projeto se torna mais real, o que gera uma ansiedade boa... e ruim também. Que comecem os jogos! \\

\protect\rule{13cm}{.5pt}
\\

\textbf{Título}: 2ª Semana do projeto

\textbf{Data da publicação}: 30 de maio de 2021

\textbf{Autora}: Mariana

Após nossa apresentação da proposta inicial no dia 25/05, ficamos mais aliviados com os \textit{feedbacks} dos professores. Um desses \textit{feedbacks} foram referentes à modelagem dos dados pertinentes da nossa aplicação, mas no fim, nossa proposta foi aceita.

Na sexta-feira, dia 29/05, nos reunimos novamente através da plataforma Google Meet para compartilharmos progressos que alguns integrantes já tinham feito, como esboços do aplicativo, ideias para apresentação de dados através de gráficos, e uma modelagem inicial dos dados. Além disso, tivemos alguns momentos de \textit{brainstorm} para trazer novas ideias e começamos a decidir quais \textit{features} poderiam ser deixadas para a versão 2.0, que seriam para o próximo semestre. Por fim, nesta reunião, decidimos que iríamos começar a fechar o escopo, ou seja, definir de fato o que vamos trabalhar nesse semestre, e começar a ter menos \textit{brainstorm} na equipe. Para que possamos concordar com o fechamento de escopo, criamos um documento no Google Docs, para listar todas as funcionalidades com o maior detalhamento possível, e para tirarmos dúvidas sobre esses detalhes. Assim, podemos começar a focar no que será entregue neste semestre! $\backslash$o/

É isso, até a próxima! \\

\protect\rule{13cm}{.5pt}
\\

\textbf{Título}: 3ª Semana do Projeto

\textbf{Data da publicação}: 8 de junho de 2021

\textbf{Autor}: Fabio

Depois de termos feito o levantamento de ideias, feito os esboços de modelagem e começado a delimitar o escopo mostramos para os professores no início dessa 3ª semana.

Os professores deram um \textit{feedback} positivo e mostraram os pontos que precisávamos rever na modelagem e escopo. Depois dessa reunião focamos em fazer as modificações necessárias para a versão final do escopo e modelagem, além de formalizar isso no documento do desenho de aplicação.

Resumindo foi uma semana bem cheia fizemos umas duas reuniões de alinhamento em equipe pelo Google Meet (uma delas com mais de duas horas de duração) e várias horas fazendo diagramas, resolvendo problemas, e pesquisando pra resolver dúvidas que apareciam. Mas no final das contas conseguimos avanços significativos.

Que venha a próxima semana!\\

\protect\rule{13cm}{.5pt}
\\

\textbf{Título}: 4ª Semana do Projeto

\textbf{Data da publicação}: 16 de junho de 2021

\textbf{Autor}: Alkindar

Nesta quarta semana de projeto ainda estávamos construindo o desenho da aplicação que foi entregue no dia 08/06. Para a construção do documento dividimos nas seguintes seções:\\



Introdução: Mariana

Planejamento do projeto: Gabi

Revisão bibliográfica: Mariana

Arquitetura: Alkindar

Escopo: Fábio 

Viabilidade financeira: Leonardo 

Escalabilidade: Alkindar

Critérios de segurança, privacidade e legislação: Fábio

Tecnologias a serem utilizadas: Carolina

Manutenibilidade da aplicação desenvolvida: Alkindar \\



Dificuldades com o \LaTeX \space a parte, diversos desdobramentos e detalhes da implementação da aplicação dos quais ainda não possuíamos de forma clara foram definidos. Infelizmente descobrimos que o modelo que utilizamos para criar o documento não era o correto, isso acarretará na correção do documento para o \textit{layout} correto. No final dessa semana também iniciamos a criação da apresentação do desenho da aplicação que foi realizada no dia 15/06.

Outros avanços obtidos essa semana:

Conseguimos configurar o Gource para gerar os vídeos do projeto. 

Organizamos os repositórios de desenvolvimento da aplicação no Github. 

A parte de autenticação do usuário no \textit{\gls{backend}} já está em desenovlvimento.

O \textit{boilerplate} da aplicação \textit{mobile} já está no repositório do \textit{mobile}\\



Estamos suando a camisa, galera. Mas bora para a próxima semana! \\

\protect\rule{13cm}{.5pt}
\\

\textbf{Título}: 5ª Semana do Projeto

\textbf{Data da publicação}: 25 de junho de 2021

\textbf{Autor}: Alkindar

Dado o \textit{feedback} menos que ideal dos professores diante do nosso desenho de projeto, optamos por incluir uma tarefa a mais na \textit{\gls{sprint}} inicial: reescrever o documento de forma a satisfazer as demandas dos professores.

Com esta nova tarefa, a equipe adotou uma estratégia de dividir e conquistar. Depois de avaliar os pontos que precisam ser melhorados do documento, uma parte da equipe focou em reescrevê-lo, enquanto outra parte encaminou o desenvolvimento necessário para a apresentação da prova de conceito, com as tarefas desta primeira \textit{\gls{sprint}}.

Focaram no documento:

\begin{itemize}
			\item Alkindar, Gabriely e Mariana.
\end{itemize}

Desenvolver as primeiras funcionalidades da aplicação:

\begin{itemize}
			\item Carolina, Fábio e Leonardo.
\end{itemize}

Seguimos cumprindo o calendário de \textit{\gls{sprint}}, realizando as cerimônias. Quando o documento for finalizado a equipe inteira focará nas funcionalidades desta \textit{\gls{sprint}}.\\

\protect\rule{13cm}{.5pt}
\\

\textbf{Título}: 6º semana do projeto

\textbf{Data da publicação}: 29 de junho de 2021

\textbf{Autora}: Gabriely

 A 6º semana foi marcada por um ritmo de desenvolvimento intenso para que, na apresentação na da \gls{poc} tudo estivesse de acordo com a expectativa alinhada pelo time. 

Para tal, a infraestrutura deveria ser construída e os times de \textit{\gls{backend}} e \textit{\gls{frontend}} deveriam trabalhar em conjunto para entregar as funcionalidades definidas para essa etapa.

Os desafios e dificuldades foram muitos, porém a cada conquista o time celebrava e comemorava. Essa semana reforçou nosso engajamento e nos mostrou desafios reais de um desenvolvimento real. Avante! \\

\protect\rule{13cm}{.5pt}
\\

\textbf{Título}: 7º Semana do Projeto

\textbf{Data da publicação}: 5 de julho de 2021

\textbf{Autor}: Leonardo

Essa semana apresentamos a \gls{poc} do projeto, apresentando o \textit{\gls{backend}}, \textit{\gls{frontend}}, infraestrutura, tecnologias e arquiteturas utilizadas. Ainda que tenham ocorrido contratempos durante a apresentação, o resultado avaliado pelos professores foi satisfatório, apontando apenas alguns pontos de melhoria na infraestrutura.

No decorrer da semana, houveram várias atividades em paralelo sendo executadas, tais como início do planejamento da documentação do projeto para a entrega no dia 13/07, o desenvolvimento inicial dos \textit{endpoints} das segunda \textit{\gls{sprint}}, o planejamento de validações de segurança na \gls{api}, a criação das primeiras telas no \textit{\gls{frontend}}, planejamento e organização de testes automatizados (tanto no \textit{\gls{backend}} quanto no \textit{\gls{frontend}}) e configuração de \gls{https} na \gls{aws}.

Faltando praticamente um mês para a finalização do semestre, estamos todos frenéticos para as entregas finais. Vamos com tudo!\\

\protect\rule{13cm}{.5pt}
\\

\textbf{Título}: 8º Semana do Projeto

\textbf{Data da publicação}: 13 de julho de 2021

\textbf{Autor}: Fabio

Essa semana foi muito corrida, com as data da entrega final se aproximando todos tiveram que acelerar o ritmo para manter tudo dentro do prazo planejado.

Seguimos com o desenvolvimento das funcionalidades que vamos implementar para a entrega do \gls{mvp} que incluem o compartilhamento das listas e o gerenciamento dos convites, na implementação dessas funcionalidades e no plano de testes. Também trabalhamos procurando soluções para alguns problemas no \textit{\gls{backend}} com a hospedagem da aplicação, mas temos tido êxito em contornar esses imprevistos.

Outra parte importante dessa semana foi continuar a compilação de tudo que foi produzido de documentação desde a primeira semana de projeto até agora no documento final que vamos entregar com o \gls{mvp}.

Não tem sido fácil conciliar trabalho, as tarefas da disciplina de Projeto Integrado e de todas as demais disciplinas, mas seguimos firmes, rumo a reta final! \\

\protect\rule{13cm}{.5pt}
\\

\textbf{Título}: 9º Semana do Projeto

\textbf{Data da publicação}: 23 de julho de 2021

\textbf{Autor}: Gabriely

A cada dia mais próximos da entrega final o grupo se encontra engajado em solucionar os pontos de desenvolvimento pendentes. Após a entrega da documentação do \gls{mvp} pudemos dedicar total energia a codificação e pesquisa de soluções que atendessem nossas necessidades e resolvessem nossos dilemas. O ritmo tem sido intenso para combinar com nossa ansiedade e a cada semana superada nos surpreendemos com como conseguimos conciliar todos os ``pratinhos'' ao mesmo tempo. Talvez, em quesito técnico, não sejamos nenhum Mark Zuckerberg, mas só por conseguirmos lidar com a vida pessoal, profissional e acadêmica no meio de uma PANDEMIA enquanto saímos das nossas zonas de conforto técnicas para entregar o melhor, somos incríveis! Fica aqui meus parabéns a cada um da equipe. Avante! \\

\protect\rule{13cm}{.5pt}
\\

\textbf{Título}: 10º Semana do Projeto

\textbf{Data da publicação}: 3 de agosto de 2021

\textbf{Autor}: Mariana

Com a chegada de final de semestre e da apresentação do nosso MVP, na última semana, estivemos focando em vários aspectos para entregar um \gls{mvp} consistente com o que propomos anteriormente. Primeiramente, nos encontramos diversas vezes durante a semana, para treinar e ensaiar a nossa apresentação através de um documento powerpoint, separando cada fala para cada integrante, além de contabilizar o tempo e tentar achar gaps que pudesse ser melhorados na apresentação final. Além disso, essa semana tivemos muita troca de suporte e ajuda, para superar desafios que surgiram de último momento. Enquanto tudo isso acontece, o documento final do mvp está sendo desenvolvido, para ser entregue na semana que vem. \\

\protect\rule{13cm}{.5pt}
\\

\textbf{Título}: 11ª Semana do Projeto -- Retomada de férias

\textbf{Data da publicação}: 29 de setembro de 2021

\textbf{Autor}: Alkindar

Nesta semana, de retomada de férias, a equipe voltou a atuar com foco e força no projeto, acelerando o ritmo em relação ao que foi desenvolvido nas férias. Durante o recesso de meio de ano, corrigimos alguns apontamentos indicados pelos professores, como 

\begin{itemize}
	\item salvar tokens de acesso no banco,
	\item alterar a terminologia: "deletar lista" para "excluir lista",
	\item confirmar a exclusão de listas, 
	\item entre outros.
\end{itemize}

Com estas tarefas em vias de finalização, e apresentadas aos professores no dia 23 de setembro, iniciamos o planejamento das funcionalidades que serão implementadas nesta segunda fase do projeto, a divisão das \textit{\glspl{sprint}} e quebra das tarefas que comporão cada uma.  Uma cronograma estimado será apresentado no próximo encontro com os professores, no dia 30 de setembro. \\

\protect\rule{13cm}{.5pt}
\\

\textbf{Título}: 12ª Semana do projeto

\textbf{Data da publicação}: 4 de outubro de 2021

\textbf{Autor}: Carolina

Durante esta semana finalizamos o planejamento e a criação do cronograma o apresentamos dia 30/09 para os professores. Conforme o \textit{feedback} da apresentação realizada nós temos itens a corrigir.

O nosso cronograma pode ser visualizado no link abaixo:

[A6PGP] Index of /S202101-PI/TGT/Documentos/Planejamento (ifsp.edu.br)

As correções solicitadas foram:

\begin{itemize}
	\item Não demarcar uma data limite para cada tarefa uma vez que isso iria de contramão ao SCRUM (metodologia que estamos utilizando);
	\item Não alocar qual recurso fará cada tarefa uma vez que no SCRUM entende-se que quem estiver disponível pode pegar a tarefa que estiver no backlog;
	\item Estabelecer o cronograma visando a finalização no dia 18/11.
\end{itemize}

Além do planejamento, prosseguimos com o desenvolvimento do \textit{\gls{backend}} e do \textit{\gls{frontend}} adiantando o início das tarefas que compõe a primeira \textit{\gls{sprint}} documentada no cronograma. \\

\protect\rule{13cm}{.5pt}
\\

\textbf{Título}: 13° Semana do projeto

\textbf{Data da publicação}: 7 de outubro de 2021

\textbf{Autor}: Fabio

Ao longo dessa semana fizemos as correções que os professores sugeriram quando apresentamos nosso planejamento até a entrega final. Além disso, continuamos com o desenvolvimento das tarefas da \textit{\gls{sprint}} atual e conseguimos concluir: 

\begin{itemize} 
	\item A melhoria do fluxo de redefinição de senha: o usuário não recebe mais uma senha aleatória e válida gerada pelo sistema, mas um link para um formulário onde ele deve inserir uma nova senha (tudo validado por tokens JWT); 
	\item A implementação dos comentários globais no front-end: Com essa funcionalidade os usuários podem acrescentar um comentário em um item que aparece em todas as listas de compras, para não ter que inserir um mesmo comentário sempre que acrescenta um item a uma nova lista;
\end{itemize}

Também tivemos nossa reunião semanal com os professores e pudemos retirar dúvidas sobre os certificados de segurança exigidos para a aplicação, mostrar as novas funcionalidades implementadas para validação e relatar as correções no planejamento. O \textit{feedback} dos professores foi positivo e as sugestões de melhorias foram muito relevantes.
Ao longo dessa semana também dedicamos um tempo para verificar o que deve ser atualizado na documentação para evitar acumular trabalho para o final do semestre.

Nessa semana foi isso. Até a próxima semana!! \\

\protect\rule{13cm}{.5pt}
\\

\textbf{Título}: 14º Semana do projeto

\textbf{Data da publicação}: 14 de outubro de 2021

\textbf{Autor}: Gabriely

 Tivemos mais uma semana superada dentro do projeto. Nela, a equipe iniciou algumas das atividades presentes em nosso backlog do Trello e também concluiu atividades que já estavam em andamento. 

\begin{itemize}
	\item Os testes de tela do \textit{\gls{frontend}}, que estavam em aberto, foram concluídos com sucesso;
	\item Foi iniciado o desenvolvimento da funcionalidade de geolocalização, para detectar o estabelecimento onde o cliente está realizando a compra; 
	\item A parte de infraestrutura passou por alterações que auxiliaram na melhoria das métricas de segurança, O próximo passo, agora, é medir o quanto isso limita a nossa base de usuários para que seja possível decidir em que ponto deixaremos a segurança; 
	\item A feature de comentário global segue sendo desenvolvida pelo time.
\end{itemize}

Também foi realizada a reunião semanal com os professores, onde pudemos compartilhar o andamento do projeto e tirar algumas dúvidas sobre as atividades. 

Estamos prontos para a próxima semana! Até a próxima \\

\protect\rule{13cm}{.5pt}
\\

\textbf{Título}: 15º Semana do Projeto

\textbf{Data da publicação}: 22 de outubro de 2021

\textbf{Autor}: Leonardo

Nessa semana, prosseguimos no desenvolvimento do projeto atuando na \textit{\gls{sprint}} 07.

\begin{itemize}
	\item Foi finalizado o fluxo base da Geolocalização no \textit{\gls{backend}} tratando erros e com testes automatizados planejados;
	\item Foi iniciada a feature de código de barras no \textit{\gls{frontend}};
	\item Tiramos as métricas do projeto (Linhas de código, commits, classes, etc);
	\item Foi continuado o processo de melhoria de metricas de segurança na infraestrutura;
	\item Houve, no dia 20/10/2021, um refining (cerimônia \textit{\gls{scrum}} para entender a regra de negócio de uma demanda) sobre os módulos de Histórico e Análise de Compras no Lixt.
\end{itemize}

Essa semana, devido ao Conselho Pedagógico para os Cursos Técnicos, não tivemos uma conexão direta com os professores orientadores do projeto. Contudo, estamos com toda energia para finalizar o projeto.

Vamos com tudo! \\

\protect\rule{13cm}{.5pt}
\\

\textbf{Título}: 16º Semana do Projeto

\textbf{Data da publicação}: 7 de novembro de 2021

\textbf{Autor}: Mariana

Na semana 16 (18/10 a 22/10), não tivemos aulas, o que nos deu a oportunidade de focar mais no desenvolvimento do Lixt. Durante a semana pudemos:

\begin{itemize}
	\item Nos reunir para discutir como poderíamos mostrar as análises de forma amigável e prática ao usuário, além da localização e histórico de compras;
	\item Discutimos sobre a funcionalidade de comentários globais, se eles deveriam ser públicos ou privados e nos marcamos de nos encontrar para discutir isso de forma mais aprofundada e técnica;
	\item No \textit{\gls{frontend}}, estávamos desenvolvendo o cadastro de produto com código de barras, além da criação de testes unitários para o leitor de código de barras.
\end{itemize}

Estamos cada dia chegando mais perto da entrega, e durante esse período, estamos tirando todas as dúvidas possíveis com os professores, além de mostrá-los como está o andamento das funcionalidades finais.

É isso! Até a próxima semana.  \\

\protect\rule{13cm}{.5pt}
\\

\textbf{Título}: 17º Semana do Projeto

\textbf{Data da publicação}: 8 de novembro de 2021

\textbf{Autor}: Alkindar

Nesta semana (23 a 30/10), demos continuidade ao desenvolvimento do projeto. Como colocado no último post, aproveitamos que na semana anterior não tivemos aula aumentar o foco no planejamento das últimas features a serem desenvolvidas, principalmente as questões de análises estatísticas, geolocalização e código de barras. 

As queries para extrair os dados, após serem definidas, começaram a ser implementadas no \textit{\gls{backend}}, e em breve estarão disponíveis no app com visualizações amigáveis para o usuário. Em paralelo, a questão dos comentários globais está avançando. 

Também no \textit{\gls{backend}} foi implementada uma flag no comentário para avisar se é um comentário público ou não, assim como um filtro, para evitar vulnerabilidades que exponham o conteúdo de todos os comentários para todos os usuários. Como ainda falta implementar a ordenação dos comentários no \textit{\gls{frontend}}, e isso requer mais flags a entidade usuário, os testes serão reescritos após finalizarmos esta feature, já que alterar a entidade usuário requer alterar todos os testes que temos. No \textit{\gls{frontend}}, a visualização básica dos comentários globais foi implementada, faltando apenas ordenar de acordo com a preferência do usuário.

Além disso, no \textit{\gls{frontend}}, seguindo recomendações dos professores, duas alterações foram feitas:

\begin{itemize}
	\item O app passa as coordenadas reais do usuário para que esta informação seja salva em banco, e o entendimento do local onde o usuário se encontra, em termos de nome e endereço é feito no \textit{\gls{backend}}.
	\item Foram feitas algumas alterações no processo de leitura do código de barras, a começar pelo ícone que indica esta funcionalidade:
		\begin{itemize}
		\item Ao cadastrar um produto já existente, o app sugere o produto para o usuário;
		\item O código de barras pode ser digitado, para casos em que a leitura via câmera não funciona (este modo de busca exigiu uma pequena mudança no \textit{\gls{backend}} também).
		\end{itemize}
\end{itemize} 

Este foi o andamento nesta última semana, estamos tão perto mas tão longe de finalizarmos... Uma hora chegamos lá \\

\protect\rule{13cm}{.5pt}
\\

\textbf{Título}: 18ª semana de projeto

\textbf{Data da publicação}: 14 de novembro de 2021

\textbf{Autor}: Carolina

Na 18ª semana de projeto, demos prosseguimento ao desenvolvimento das tarefas do cronograma além dos ajustes solicitados pelos professores nas conversas semanais.

Esta semana foi completada a funcionalidade de alteração do status (privado ou público) do comentário global, correções na interface de comentários e testes da mesma. Além disso, a tela de histórico de compras também foi desenvolvida e testada.

No \textit{\gls{backend}} ajustes dos comentários globais também foram realizados: filtragem dos comentários globais no servidor para evitar vulnerabilidade e implementação da flag de para preferência de ordenação. Além disso, foi dado prosseguimento nas tarefas relativas às queries da funcionalidade de estatísticas.

Quanto à documentação, foram realizadas revisões e correções do documento para a adequação ao modelo proposto na disciplina.

Este foi o progresso da 18ª semana de projeto. Estamos chegando ao fim e se tudo der certo significa que estamos próximos dos últimos relatórios de progresso também

Até mais!