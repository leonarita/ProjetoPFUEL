%% Adaptado a partir de :
%%    abtex2-modelo-trabalho-academico.tex, v-1.9.2 laurocesar
%% para ser um modelo para os trabalhos no IFSP-SPO

\documentclass[
    % -- opções da classe memoir --
    12pt,               % tamanho da fonte
    openright,          % capítulos começam em pág ímpar (insere página vazia caso preciso)
    %twoside,            % para impressão em verso e anverso. Oposto a oneside
    oneside,
    a4paper,            % tamanho do papel. 
    % -- opções da classe abntex2 --schwinn
    % Opções que não devem ser utilizadas na versão final do documento
    %draft,              % para compilar mais rápido, remover na versão final
    paginasA3,  % indica que vai utilizar paginas em A3 
    MODELO,             % indica que é um documento modelo então precisa dos geradores de texto
    TODO,               % indica que deve apresentar lista de pendencias 
    % -- opções do pacote babel --
    english,            % idioma adicional para hifenização
    brazil              % o último idioma é o principal do documento
    ]{ifsp-spo-inf-ctds} % ajustar de acordo com o modelo desejado para o curso


% ---
% Informações de dados para CAPA e FOLHA DE ROSTO
% ---
\titulo{Análise Exploratória de Dados (EDA) \\ Estudo de Caso}

% Trabalho individual
%\autor{AUTOR DO TRABALHO}

% Trabalho em Equipe
% ver também https://github.com/abntex/abntex2/wiki/FAQ#como-adicionar-mais-de-um-autor-ao-meu-projeto
\renewcommand{\imprimirautor}{
\begin{tabular}{lr}
Leonardo Naoki Narita & SP3022498 
\end{tabular}
}


\disciplina{PFUEL - Programação Funcional}

\preambulo{Trabalho desenvolvido no curso de Tecnologia
em Análise e Desenvolvimento de Sistemas
do Instituto Federal de Educação, Ciência e
Tecnologia de São Paulo como requisito parcial
para a conclusão da disciplina de Programação Funcional.}

\data{27 de Janeiro de 2022}

% Definir o que for necessário e comentar o que não for necessário
% Utilizar o Nome Completo, abntex tem orientador e coorientador
% então vão ser utilizados na definição de professor
\renewcommand{\orientadorname}{Professor:}
\orientador{Guilherme Werneck de Oliveira}


% ---


% informações do PDF
\makeatletter
\hypersetup{
        %pagebackref=true,
        pdftitle={\@title}, 
        pdfauthor={\@author},
        pdfsubject={\imprimirpreambulo},
        pdfcreator={LaTeX with abnTeX2 using IFSP model},
        pdfkeywords={abnt}{latex}{abntex}{abntex2}{IFSP}{\ifspprefixo}{trabalho acadêmico}, 
        colorlinks=true,            % false: boxed links; true: colored links
        linkcolor=blue,             % color of internal links
        citecolor=blue,             % color of links to bibliography
        filecolor=magenta,              % color of file links
        urlcolor=blue,
        bookmarksdepth=4
}
\makeatother
% --- 

\graphicspath{ {./images} }
\usepackage{float}

% carregando aqui referencias quando utilizando BIBLATEX
\IfPackageLoaded{biblatex}{%
\addbibresource{referencias.bib}
\addbibresource{exemplos/abntex2-doc-abnt-6023.bib}
}{}

% ----
% Início do documento
% ----
\begin{document}


% Retira espaço extra obsoleto entre as frases.
\frenchspacing 

% -- lista de pendencias gerada pelo todonotes
% -- altere opções do usepackage para remover na versão final....
%\listoftodos
%\todo[inline]{remover lista de todo da versão final...}

\newpage

% ----------------------------------------------------------
% ELEMENTOS PRÉ-TEXTUAIS
% ----------------------------------------------------------
\pretextual

% ---
% Capa
% ---
\imprimircapa
\newcounter{todocounter}
\newcommand{\todonum}[2][]
{\stepcounter{todocounter}\todo[#1]{\thetodocounter: #2}}


% ---

% ---
% Folha de rosto
% (o * indica que haverá a ficha bibliográfica)
% ---
\imprimirfolhaderosto

%\todo[inline]{colocar a data de entrega}

%\imprimirfolhaderosto*
% ---

% Quando registrado na biblioteca
%\input{pre-fichacatalografica}

%Caso necessário
%\input{pre-errata}

%Obrigatório para trabalhos com bancas oficiais
%\input{pre-aprovacao}


% ---
% inserir lista de ilustrações
% ---
%\pdfbookmark[0]{\listfigurename}{lof}
%\listoffigures*
%\cleardoublepage
% ---

% ---
% inserir lista de tabelas
% ---
%\pdfbookmark[0]{\listtablename}{lot}
%\listoftables*
%\cleardoublepage
% ---

% ---
% inserir lista de quadros
% ---
%\pdfbookmark[0]{\listofquadrosname}{loq}
%\listofquadros*
%\cleardoublepage
% ---


% ---
% inserir o sumario
% ---
\pdfbookmark[0]{\contentsname}{toc}
\tableofcontents*
\cleardoublepage
% ---


% ----------------------------------------------------------
% ELEMENTOS TEXTUAIS
% ----------------------------------------------------------
\textual

\chapter{Introdução}	

Nesse capítulo, são abordados conceitos fundamentais que embasam o desenvolvimento do projeto.

\section{Definição de EDA}

A Análise Exploratória de Dados (do inglês, "Exploratory Data Analysis", conhecida também pela sigla "EDA") é a aplicação de um conjunto de técnicas que visam analisar uma população ou amostra de dados (conjunto de dados dentro de um mesmo contexto).

A EDA visa identificar conclusões significativas dentro dessa massa de dados, podendo ser padrões ou outliers (inconsistências de dados).

\section{Ciclo de Vida da EDA}

Dado que há uma massa de dados importada e devidamente tratada (tendo-os consistentes), o ciclo de vida da EDA baseia-se em três passos:

\subsection{Gerar Questionamentos}

O primeiro passo é gerar questionamentos sobre os dados que estão dispostos, criando variáveis e agregações.

Para gerar questionamentos, uma questão é fundamental a ser indagada: "Que tipo de variação ocorre com as variáveis?".

\subsection{Modelas Dados}

O segundo passo é organizar os dados de modo que seja visualmente fácil de analisar para que, assim possa encontrar conclusões.

\subsection{Buscar Conclusões}

O terceiro e último passo é analisar as conclusões obtidas, podendo chegar a novos questionamentos.






\chapter{Apresentação do Conjunto de Dados}	

Nesse capítulo, é definida a massa de dados a ser importado no projeto, bem como suas variáveis e detalhes.

\section{Escolha dos Dados}

O conjunto de dados escolhidos para o projeto é o "Brazilian Cites", onde há 5573 cidades brasileiras, disponível em: https://www.kaggle.com/crisparada/brazilian-cities.

\section{Dicionário de Dados}

Nessa seção, é definido os principais atributos que serão trabalhados no EDA. 

Os principais atributos das cidades do Brasil, dispostas na planilha, são:

\begin{quadro}[H]
\centering
\ABNTEXfontereduzida
\caption[Definição dos Principais Atributos]{Definição dos Principais Atributos}
\label{dicionario-dados-globalcomment}
\begin{tabular}{|p{5.0cm}|p{5.0cm}|p{5.0cm}|}

  \hline
  \multicolumn{3}{|c|}{Cidades do Brasil} \\
  
  \hline
  \thead{Atributo} & \thead{Descrição}  & \thead{Valores} \\
    
  \hline 
  CITY & Nome da cidade &  \\
  
  \hline 
  STATE & Nome do estado &  \\

  \hline 
  CAPITAL & Indica se a cidade é a capital do estado & 1 (SIM) ou 0 (NÃO) \\
  
  \hline 
  IBGE\_RES\_POP & População residente na cidade &  \\
  
  \hline 
  IBGE\_RES\_POP\_BRAS & População brasileira residente na cidade &  \\
  
  \hline 
  IBGE\_RES\_POP\_ESTR & População estrangeira residente na cidade &  \\
  
  \hline 
  IDHM & Índice de Desenvolvimento Humano (IDH) &  \\
  
  \hline 
  IDHM\_Renda & Índice de renda pelo IDH &  \\
  
  \hline 
  IDHM\_Longevidade & Índice de longevidade pelo IDH &  \\
  
  \hline 
  IDHM\_Educacao & Índice de educação pelo IDH &  \\
  
  \hline
\end{tabular}
\legend{Fonte: \href{https://www.kaggle.com/crisparada/brazilian-cities?select=Data_Dictionary.csv}{Kaggle}}
\end{quadro}


\chapter{Desenvolvimento da EDA}

Nesse capítulo, é aplicado os três passos do ciclo de uma EDA em um contexto prático aplicando o conjunto de dados "Brazilian Cities".

\section{Contextualização}

O IDH (Índice de Desenvolvimento Humano) é um índice que avalia a qualidade de vida em um determinado local, sendo mensurado a partir de 3 fatores: Longevidade, Educação e Renda. No Brasil, o órgão responsável por avaliar o IDH é o IBGE (Instituto Brasileiro de Geografia e Estatística).

Sendo um parâmetro importante, o IDH é capaz de influenciar tomadas de decisões, tais como quais setores aplicar investimento público pelos políticos, se é vantajoso mudar-se para morar nesse local, e se é viável a iniciativa privada investir nesse local para atuar.

Vide na figura \autoref{fig:distribuicao-idh-nas-cidades} a distribuição do IDH nas cidades do Brasil.

\begin{figure}[H]
  \centering
  \caption{\label{fig:distribuicao-idh-nas-cidades}Distribuição do IDH nas cidades brasileiras}
  \label{fig:der}
  \includegraphics[scale=0.5]{chapter-03-case-01-img-01}
  \legend{Fonte: O autor}
\end{figure}

Visualizando esse gráfico, é perceber que não há cidades brasileiras com IDH menor do que 0.4, nem com IDH maior do que 0.8.

\section{Tratamento de Dados}

Para tornar os dados mais consistentes, foi necessário filtrar os dados cujo IDH não seja NA (Não Aplicável).

\section{Problematização}

Com isso, surgem-se duas questões: 

\begin{enumerate}
	\item "Qual é a razão da discrepância entre os dados de IDH?";
	\item "Por que não existe IDH maior que 1?".
\end{enumerate}

Para responder essas questões, foram feitas análises da variação de cada critério do IDH. Na \autoref{fig:idh-renda}, é analisada a variação de renda. Na \autoref{fig:idh-longevidade}, é analisada a variação de longevidade. E, por fim, na figura \autoref{fig:idh-educacao}, é analisada a variação de educação.

\begin{figure}[H]
  \centering
  \caption{\label{fig:idh-renda}Distribuição do IDH de Renda nas cidades brasileiras}
  \label{fig:der}
  \includegraphics[scale=0.5]{chapter-03-case-01-img-02}
  \legend{Fonte: O autor}
\end{figure}

\begin{figure}[H]
  \centering
  \caption{\label{fig:idh-longevidade}Distribuição do IDH de Longevidade nas cidades brasileiras}
  \label{fig:der}
  \includegraphics[scale=0.5]{chapter-03-case-01-img-03}
  \legend{Fonte: O autor}
\end{figure}

\begin{figure}[H]
  \centering
  \caption{\label{fig:idh-educacao}Distribuiçãa do IDH de Educação nas cidades brasileiras}
  \label{fig:der}
  \includegraphics[scale=0.5]{chapter-03-case-01-img-04}
  \legend{Fonte: O autor}
\end{figure}

Dado os gráficos acima, foram realizados cálculos para conseguir realizar conclusões mais assertivas. Foram realizados cálculos de média simples e desvio padrão (desigualdade entre os dados, sendo 0 mais próximo da igualdade e 1 mais próximo da desigualdade).

Calculando a média de cada critério do IDH, foi constatado que a renda teve a nota 0.65, a longevidade teve a nota 0.80 e a educação teve a nota 0.55.

Agora calculando o desvio padrão de cada critério do IDH, foi constatado que a renda teve a nota 0.08, a longevidade teve a nota 0.04 e a educação teve a nota 0.09.



\chapter{Conclusão}

Nesse projeto, foi-se iniciado por definições de EDA, seus métodos e definições básicas. Logo após, continuando com a importação de dados, definindo os principais atributos disponíveis na massa de dados disponibilizados para que, com isso, seja possível trabalhar com a metodologia de EDA por completa.

Com isso, geramos questionamentos iniciais e buscamos visualizar os dados a fim de respondê-las. Nessa seção, iremos concluir os raciocínios e analisar se foi possível responder aos questionamentos ou se serão formados novos questionamentos.

O primeiro caso analisado ...

O segundo caso analisado, por sua vez, ...



% ----------------------------------------------------------
% Finaliza a parte no bookmark do PDF
% para que se inicie o bookmark na raiz
% e adiciona espaço de parte no Sumário
% ----------------------------------------------------------
\phantompart

% ----------------------------------------------------------
% ELEMENTOS PÓS-TEXTUAIS
% ----------------------------------------------------------
\postextual
% ----------------------------------------------------------

% ----------------------------------------------------------
% Referências bibliográficas
% ----------------------------------------------------------
% quando não esta utilizando biblatex tem que carregar as referencias aqui

%
%\IfPackageLoaded{biblatex}{}{%
%\bibliography{referencias/referencias}
%}

%\input{pos-anexos}

%---------------------------------------------------------------------

\end{document}