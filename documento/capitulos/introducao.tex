\chapter{Introdução}	

A Análise Exploratória de Dados (do inglês, "Exploratory Data Analysis", conhecida também pela sigla "EDA") é a aplicação de um conjunto de técnicas que visam analisar uma população ou amostra de dados (conjunto de dados dentro de um mesmo contexto).

A EDA visa identificar conclusões significativas dentro dessa massa de dados, podendo ser padrões ou outliers (inconsistências de dados).

O ciclo de vida da EDA baseia-se em três passos contínuos:

\begin{enumerate}
	\item Gerar questionamentos sobre os dados que possui;
	\item Transformar e modelar os dados a fim de encontrar repostas;
	\item A partir das conclusões, gerar novos questionamentos.
\end{enumerate}	


