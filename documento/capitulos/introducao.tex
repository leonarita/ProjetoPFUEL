\chapter{Introdução}	

Nesse capítulo, são abordados conceitos fundamentais que embasam o desenvolvimento do projeto.

\section{Definição de EDA}

A Análise Exploratória de Dados (do inglês, "Exploratory Data Analysis", conhecida também pela sigla "EDA") é a aplicação de um conjunto de técnicas que visam analisar uma população ou amostra de dados (conjunto de dados dentro de um mesmo contexto).

A EDA visa identificar conclusões significativas dentro dessa massa de dados, podendo ser padrões ou outliers (inconsistências de dados).

\section{Ciclo de Vida da EDA}

Dado que há uma massa de dados importada e devidamente tratada (tendo-os consistentes), o ciclo de vida da EDA baseia-se em três passos:

\subsection{Gerar Questionamentos}

O primeiro passo é gerar questionamentos sobre os dados que estão dispostos, criando variáveis e agregações.

Para gerar questionamentos, uma questão é fundamental a ser indagada: "Que tipo de variação ocorre com as variáveis?".

\subsection{Modelas Dados}

O segundo passo é organizar os dados de modo que seja visualmente fácil de analisar para que, assim possa encontrar conclusões.

\subsection{Buscar Conclusões}

O terceiro e último passo é analisar as conclusões obtidas, podendo chegar a novos questionamentos.




