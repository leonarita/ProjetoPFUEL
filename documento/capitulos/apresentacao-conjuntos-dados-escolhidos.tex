\chapter{Apresentação do Conjunto de Dados}	

Nesse capítulo, é definida a massa de dados a ser importado no projeto, bem como suas variáveis e detalhes.

\section{Escolha dos Dados}

O conjunto de dados escolhidos para o projeto é o "Brazilian Cites", onde há 5573 cidades brasileiras, disponível em: https://www.kaggle.com/crisparada/brazilian-cities.

\section{Dicionário de Dados}

Nessa seção, é definido os principais atributos que serão trabalhados e analisados pela EDA. 

Os principais atributos das cidades do Brasil, dispostas na planilha, são:

\begin{quadro}[H]
\centering
\ABNTEXfontereduzida
\caption[Definição dos Principais Atributos]{Definição dos Principais Atributos}
\label{dicionario-dados-globalcomment}
\begin{tabular}{|p{5.0cm}|p{5.0cm}|p{5.0cm}|}

  \hline
  \multicolumn{3}{|c|}{Cidades do Brasil} \\
  
  \hline
  \thead{Atributo} & \thead{Descrição}  & \thead{Valores} \\
    
  \hline 
  CITY & Nome da cidade &  \\
  
  \hline 
  STATE & Nome do estado &  \\

  \hline 
  CAPITAL & Indica se a cidade é a capital do estado & 1 (SIM) ou 0 (NÃO) \\
  
  \hline 
  IBGE\_RES\_POP & População residente na cidade &  \\
  
  \hline 
  IBGE\_RES\_POP\_BRAS & População brasileira residente na cidade &  \\
  
  \hline 
  IBGE\_RES\_POP\_ESTR & População estrangeira residente na cidade &  \\
  
  \hline 
  IDHM & Índice de Desenvolvimento Humano (IDH) &  \\
  
  \hline 
  IDHM\_Renda & Índice de renda pelo IDH &  \\
  
  \hline 
  IDHM\_Longevidade & Índice de longevidade pelo IDH &  \\
  
  \hline 
  IDHM\_Educacao & Índice de educação pelo IDH &  \\
  
  \hline
\end{tabular}
\legend{Fonte: \href{https://www.kaggle.com/crisparada/brazilian-cities?select=Data_Dictionary.csv}{Kaggle}}
\end{quadro}

Na coluna "Valores", as células não preenchidas representam variáveis contínuas (conjunto de valores abertos, ou seja, possuindo uma quantidade de variáveis que são variáveis). Enquanto, nessa mesma colunas, as células preenchidas representam variáveis categóricas (com um conjunto de dados fechados e pré-determinados).

