\chapter{Desenvolvimento da EDA}

Nesse capítulo, é aplicado os três passos do ciclo de uma EDA em um contexto prático aplicando o conjunto de dados "Brazilian Cities".

\section{Caso 1: Análise do IDH}

O IDH (Índice de Desenvolvimento Humano) é um índice que avalia a qualidade de vida em um determinado local, sendo mensurado a partir de 3 fatores: Longevidade, Educação e Renda. No Brasil, o órgão responsável por avaliar o IDH é o IBGE (Instituto Brasileiro de Geografia e Estatística).

Sendo um parâmetro importante, o IDH é capaz de influenciar tomadas de decisões, tais como quais setores aplicar investimento público pelos políticos, se é vantajoso mudar-se para morar nesse local, e se é viável a iniciativa privada investir nesse local para atuar.

Com isso, surge-se uma questão: "Será que as cidades mais populosas são as que possuem o maior IDH?".

...

\section{Caso 2: Análise de Estrangeiros}

No mesmo contexto do caso 1, outra questão deve ser feita: "Um estrangeiro, de modo geral, tende a ir à cidades com maiores índices de IDH ou cidades mais populosas para morar?".

...




