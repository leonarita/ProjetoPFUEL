\chapter{Conclusão}

Nesse projeto, foi-se iniciado por definições de EDA, seus métodos e definições básicas. Logo após, continuando com a importação de dados, definindo os principais atributos disponíveis na massa de dados disponibilizados para que, com isso, seja possível trabalhar com a metodologia de EDA por completa.

Com isso, geramos questionamentos iniciais e buscamos visualizar os dados a fim de respondê-las. Nessa seção, iremos concluir os raciocínios e analisar se foi possível responder aos questionamentos ou se serão formados novos questionamentos.

Analisando os gráficos fornecidos, percebe-se que a discrepância de IDH é devido, principalmente, à variação de educação, que possui a menor média e o maior desvio padrão entre os estados, resultando no maior grau de variação entre as cidades do Brasil (entre 0.3 e 0.8).

Também é possível responder que não há dados de IDH maiores que 1, visto que é uma padronização a nível global.
